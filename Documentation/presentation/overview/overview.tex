\subsection{Galaxy Clusters, Super Massive Black Holes \& X-Ray Flares}
\begin{frame}
\frametitle{What Are Galaxy Clusters?}
\end{frame}

\begin{frame}[allowframebreaks]
\frametitle{What Are Super Massive Black Holes?}
\begin{center}
\emph{Isn't that a song by Muse?}
\end{center}
\begin{itemize}
\item \alert{Super Massive Black Holes (SMBHs) are enormous gavitational singularities found at the center of galaxies} 
\\
{\small \ldots also, Muse sings about them for some reason} 

\item There is one in our galaxy named Sgr A$^\ast$\\
(pronounced: ``Sagittarius Aye Star'')\footnote{Source: \url{http://rsd-www.nrl.navy.mil/7213/lazio/GC/GC-P-BCD.nolabel-highres.jpg}} 
\end{itemize}

\framebreak

\begin{figure}[h]
{\centering
	\mode<presentation>{
	\includegraphics[width=0.8\textwidth]{images/Sgr-A-star.jpg}
	%\caption{}
	}
	\mode<article>{
	\includegraphics[width=0.5\textwidth]{images/Sgr-A-star.jpg}
	\caption{Sgr A$^\ast$ in Radio (330 MHz).  FIXME: citation for http://rsd-www.nrl.navy.mil/7213/lazio/GC/}
	}
}
\end{figure}
\end{frame}

\begin{frame}[allowframebreaks]
\frametitle{What Are X-Ray Flares?}
\begin{itemize}
	\item Ordinarily a \emph{mild mannered} SMBH, such as the Sgr A$^\ast$, has some various objects in orbit around it (see Fig \ref{Sgr-A-star-orbits})
	\item When one of these objects passes \emph{too close} to the SMBH, some gas is pulled off of it
	\begin{itemize}
		\item The matter is accellerated to extreme velocities, approaching the speed of light
		\item Lets off X-Ray light -- an \emph{X-Ray flare}
	\end{itemize}
	\item FIXME: Discuss accretion disks, jets
	\item FIXME: Cite recent Swift paper in \emph{Science} that just found one of these things
\end{itemize}
\framebreak
%%% FIXME Get animation working?
%\mode<presentation>{\animategraphics[autoplay,loop,height=\textheight]{12}{images/Sgr-A-star-orbits/Sgr-A-star-orbits-}{0}{80}}
\begin{figure}[h]
{\centering
	\mode<presentation>{
	\includegraphics[height=0.8\textheight]{images/Sgr-A-star-orbits.png}
	\caption{}
	}
	\mode<article>{
	\includegraphics[width=0.5\textwidth]{images/Sgr-A-star-orbits.png}
	\caption{Orbits of various young stars around Sgr A$^\ast$, 1995--2010.  FIXME: citation for http://www.astro.ucla.edu/~ghezgroup/gc/publications/orbitsApJ.shtml, http://www.astro.ucla.edu/~ghezgroup/gc/pictures/orbitsOverImage10.shtml}
	}
	\label{Sgr-A-star-orbits}
}
\end{figure}
\end{frame}

\subsection{How Do Astronomers Analyze Images?}
\begin{frame}[allowframebreaks]
\frametitle{Traditional Image Processing \\
vs. High Energy Astronomy}
High energy astronomers \alert{do not} use exactly the same concepts as traditional image processing
%\\~\\ 
%Instead, they have their own \emph{analogues}: 
\begin{center}
  \begin{minipage}{0.75\textwidth}
\begin{block}{\centering Image Processing $\Leftrightarrow$ X-Ray Astronomy}
\begin{center}
      \begin{tabular}{ccccc}
      $\langle x, y\rangle$ & \ \ \ & $\langle R, G, B\rangle$ & \ \ \ & $\alpha$\\
      $\Updownarrow$  & & $\Updownarrow$  &  & $\Updownarrow$ \\
      $\langle \phi, \delta\rangle$ &  & \textup{keV} &  & \textup{photon counts}
      \end{tabular}
\end{center}
\end{block}
\end{minipage}
\end{center}

\framebreak

Where:
\begin{itemize}
\item $\langle \phi,\delta\rangle$ are \emph{spherical coordinates}
	\begin{itemize}
	\item $\phi$ is latitude
	\item $\delta$ is longitude
	\end{itemize}
\item keV are \emph{kilo-electron-Volts}; 1 keV $\approx 1.6\times10^{-16}$ joules
	\begin{itemize}
	\item High energy astronomers look at $0.2$ keV -- $10$ keV
	\item In terms of wavelength ($\lambda$), the range is $6.2$ nm to $124$ pm
	\item By comparison, optical light ranges from $390$ nm (violet) to $750$ nm (red) -- that is from $0.0016$ keV to $0.0031$ keV
	\end{itemize}
\item The ``photon count'' is a direct measure of how many photons were detected by a CCD instrument during an exposure
\end{itemize}
\end{frame}
\subsection{Celestial Coordinate Systems}
\begin{frame}
\frametitle{Celestial Coordinate Systems}
Astronomers commonly use of \alert{three} celestial coordinate systems:
\begin{itemize}
\item Equatorial Coordinates
\item Ecliptic Coordinates
\item Galactic Coordinates
\end{itemize}
\end{frame}
\begin{frame}
\frametitle{Equatorial Coordinates}
\begin{columns}[c]
\column{0.6\textwidth}
\begin{itemize}
\item $\langle\phi,\delta\rangle$ determined by the Earth's axis of rotation (just like lat/lon)
	\begin{itemize}
	\item $\phi$ called \emph{Right Ascension} (RA)
	\item $\delta$ called \emph{Declination} (DEC)
	\end{itemize}
\item Under the \emph{J2000} spec: RA=0, DEC=0 set to the \emph{vernal equinox} (\Aries), where the \emph{ecliptic plane} swept by the sun and the equatorial plane intersected in the year 2000 (see Fig. \ref{equatorial-fig})\footnote{Source: \url{http://upload.wikimedia.org/wikipedia/commons/c/c2/Equatorial_coordinate_system_\%28celestial\%29.svg}}
\end{itemize}
\column{0.4\textwidth}
\begin{figure}[h]
{\centering
\mode<presentation>{
	\includegraphics[width=\textwidth]{images/Equatorial-coordinate-system}
	\caption{}
}
\mode<article>{
	\includegraphics[width=.5\textwidth]{images/Equatorial-coordinate-system}
	\caption{In the Equatorial coordinate system, 0 longitude is defined by the Earth's Equator}
}
\label{equatorial-fig}
}
\end{figure}

\end{columns}
\end{frame}
\begin{frame}
\frametitle{Ecliptic Coordinates}
\begin{columns}[c]
\column{0.6\textwidth}
\begin{itemize}
\item $\langle\phi,\delta\rangle$ determined by the Earth/Sun orbital plane
	\begin{itemize}
	\item $\phi$ labeled $\lambda$ 
	\item $\delta$ labeled $\beta$
	\end{itemize}
\item Under the \emph{J2000} spec: RA=0, DEC=0 set to the \emph{vernal equinox} (\Aries), same as equatorial (see Fig. \ref{ecliptic-figure})\footnote{Source: \url{http://upload.wikimedia.org/wikipedia/commons/b/b3/Ecliptic_coordinate_system_\%28celestial\%29.svg}}
\end{itemize}
\column{0.4\textwidth}
\begin{figure}[h]
{\centering
\mode<presentation>{
\includegraphics[width=\textwidth]{images/Ecliptic-coordinate-system}
\caption{}
}
\mode<article>{
\includegraphics[width=.5\textwidth]{images/Ecliptic-coordinate-system}
\caption{In the Ecliptic coordinate system, 0 longitude is defined by the Earth-Sun orbital plane}
}
\label{ecliptic-figure}
}
\end{figure}
\end{columns}
\end{frame}
\begin{frame}
\frametitle{Galactic Coordinates}
Galactic coordinates determined by the plane of the Milky Way\footnote{Source: \url{http://upload.wikimedia.org/wikipedia/commons/6/60/ESO_-_Milky_Way.jpg}}\\
Origin is the center of the galaxy (roughly the position of Sgr A$^\ast$)
\begin{figure}[h]
{\centering
\mode<presentation>{
	\includegraphics[width=0.8\textwidth]{images/Milky-Way-Galactic-Plane}
}
\mode<article>{
	\includegraphics[width=.5\textwidth]{images/Milky-Way-Galactic-Plane}
	\caption{A Mullweide projection of the Milky Way; created with the \texttt{matplotlib.basemap} python module}
}
}
\end{figure}
{\centering $\phi$ labeled $\ell$, $\delta$ labeled $b$}
\end{frame}
\begin{frame}
\frametitle{Conventions \& Conversion}
\begin{itemize}
\item Conventions
	\begin{itemize}
	\item In \emph{Equatorial}, RA is denoted in \emph{Hours}:\emph{Minutes}:\emph{Seconds},\\
	 DEC is denoted is \emph{Degrees}:\emph{Minutes}:\emph{Seconds}
	\item In Ecliptic/Galactic, all coordinates are denoted \emph{Degrees}:\emph{Minutes}:\emph{Seconds}\\
	$\Longrightarrow$ Just like familiar lattitude/longitude coordinates for GPS
	\end{itemize}
\item Conversion
	\begin{itemize}
	\item We use \texttt{pyephem}, an industry-standard python module in astronomy:\\
	\url{http://rhodesmill.org/pyephem/}
	\end{itemize}
\end{itemize}
\end{frame}
